\documentclass[10pt]{article}

\usepackage[utf8]{inputenc}
\usepackage{latexsym,amsfonts,amssymb,amsthm,amsmath}
\setlength{\parindent}{0in}
\setlength{\parskip}{\baselineskip}
\setlength{\oddsidemargin}{0in}
\setlength{\textwidth}{6.5in}
\setlength{\textheight}{8.8in}
\setlength{\topmargin}{0in}
\setlength{\headheight}{18pt}

\usepackage[a4paper,margin=1in,footskip=0.25in]{geometry}

\usepackage{listings}
\usepackage{color} %red, green, blue, yellow, cyan, magenta, black, white
\definecolor{mygreen}{RGB}{28,172,0} % color values Red, Green, Blue
\definecolor{mylilas}{RGB}{170,55,241}

\usepackage{graphicx}
\graphicspath{{../writeup/}}

\def\code#1{\texttt{#1}}

\usepackage[colorlinks=true, urlcolor=blue, linkcolor=red]{hyperref}

\title{PHYS 410 Project 1}
\author{Gavin Pringle, 56401938}

%%%%%%%%%%%%%%%%%%%%%%%%%%%%%%%%%%%%%%%%%%%%%%%%%%%%%%%%%%%%%%%%%%%%%%%%%%%%%%%%%%%%%%%%%%%%%%%%%%%%%%%
% Start of document
%%%%%%%%%%%%%%%%%%%%%%%%%%%%%%%%%%%%%%%%%%%%%%%%%%%%%%%%%%%%%%%%%%%%%%%%%%%%%%%%%%%%%%%%%%%%%%%%%%%%%%%
\begin{document}

\maketitle

\lstset{language=Matlab,%
    %basicstyle=\color{red},
    breaklines=true,%
    morekeywords={matlab2tikz},
    keywordstyle=\color{blue},%
    morekeywords=[2]{1}, keywordstyle=[2]{\color{black}},
    identifierstyle=\color{black},%
    stringstyle=\color{mylilas},
    commentstyle=\color{mygreen},%
    showstringspaces=false,%without this there will be a symbol in the places where there is a space
    numbers=left,%
    numberstyle={\tiny \color{black}},% size of the numbers
    numbersep=9pt, % this defines how far the numbers are from the text
    emph=[1]{for,end,break},emphstyle=[1]\color{red}, %some words to emphasise
    %emph=[2]{word1,word2}, emphstyle=[2]{style},    
}

%%%%%%%%%%%%%%%%%%%%%%%%%%%%%%%%%%%%%%%%%%%%%%%%%%%%%%%%%%%%%%%%%%%%%%%%%%%%%%%%%%%%%%%%%%%%%%%%%%%%%%%
% Introduction
%%%%%%%%%%%%%%%%%%%%%%%%%%%%%%%%%%%%%%%%%%%%%%%%%%%%%%%%%%%%%%%%%%%%%%%%%%%%%%%%%%%%%%%%%%%%%%%%%%%%%%%
\subsection*{Introduction}

In this project, the problem of $N$ identical point charges confined to free motion on the surface of a 
sphere is examined. Via a finite-difference approximation simulation, the dynamic behaviour of point 
charges each originating from random initial positions on the surface of the sphere is computed. 
Since like charges repel, if a velocity-dependent retarding force is present for each charge then it 
follows that all charges will eventually reach a stable equilibrium where the electrostatic potential
energy is minimized. 
 
Through examining the potential energy of these equilibrium configurations as well as conducting 
equivalence class analysis, the equilibrium configurations of the charges are cataloged and their
symmetry is characterized. The equilibrium configurations in this problem are described in detail 
in the \href{https://en.wikipedia.org/wiki/Thomson_problem}{Thomson problem}.

In order to test the validity of the numerical model constructed for this problem, convergence testing 
is also applied. This is done by simulating an identical scenario using multiple discretization levels,
and analyzing the level-to-level differences. This convergence testing allows us to determine to which
order our finite difference approximation is accurate.

\pagebreak

%%%%%%%%%%%%%%%%%%%%%%%%%%%%%%%%%%%%%%%%%%%%%%%%%%%%%%%%%%%%%%%%%%%%%%%%%%%%%%%%%%%%%%%%%%%%%%%%%%%%%%%
% Review of Theory
%%%%%%%%%%%%%%%%%%%%%%%%%%%%%%%%%%%%%%%%%%%%%%%%%%%%%%%%%%%%%%%%%%%%%%%%%%%%%%%%%%%%%%%%%%%%%%%%%%%%%%%
\subsection*{Review of Theory}

\subsubsection*{Equations of motion}

In this simulation, natural units are used for all variables. This allows equations to be simplified by
setting all masses and charges as well as the radius of the confining sphere centered at the origin
equal to 1: 
$$m_i = 1, \quad q_i = 1, \quad R = 1$$
Using Cartesian coordinates, the position of each charge is written as 
$$\mathbf{r}_i(t) \equiv [x_i(t), y_i(t), z_i(t)] \quad , \quad i = 1, 2, \dots, N,$$
where
$$r_i \equiv |\mathbf{r}_i| \equiv \sqrt{x_i^2 + y_i^2 + z_i^2} = 1 \quad , \quad i = 1, 2, \dots, N.$$
The separation vectors between charges can be computed using the following formulas:
$$\mathbf{r}_{ij} = \mathbf{r}_j - \mathbf{r}_i \,$$
$$r_{ij} = |\mathbf{r}_j - \mathbf{r}_i| \,$$
$$\hat{r}_{ij} \equiv \frac{\mathbf{r}_j - \mathbf{r}_i}{r_{ij}} = \frac{\mathbf{r}_{ij}}{r_{ij}} \,.$$
The variable $\gamma$ is used as the scaling parameter for the velocity-dependent retarding force. The 
equation for Newton's second law can then be written for each charge as:
$$m_i a_i = F_{i, \textrm{electrostatic}} + F_{i, \textrm{retarding}}$$
which can be expanded to 
$$m_i \mathbf{a}_i=-k_e \sum_{j=1, j \neq i}^N \frac{q_i q_j}{r_{i j}} 
\hat{\mathbf{r}}_{i j}-\gamma \mathbf{v}_i, \quad i=1,2, \ldots N, \quad 0 \leq t \leq t_{\max }.$$
Using natural units ($k_e=1$) and writing $\mathbf{a}_i$ and $\mathbf{v}_i$ as derivatives of $\mathbf{r}_i$, 
this expression can be simplified to 
$$\frac{d^2 \mathbf{r}_i}{d t^2}=-\sum_{j=1, j \neq i}^N \frac{\mathbf{r}_{i j}}{r_{i j}{ }^3}-\gamma 
\frac{d \mathbf{r}_i}{d t}, \quad i=1,2, \ldots N, \quad 0 \leq t \leq t_{\max }.$$
The above equation is what is used to numerically solve the equations of motion using FDAs. 

\subsubsection*{Electrostatic Potential Energy}

The electrostatic potential energy of a point charge distribution is given by the following formula:
$$W=k_e \sum_{i=1}^N \sum_{j>i}^N \frac{q_i q_j}{r_{i j}}$$
Writing this potential energy in natural units and as a function of time, we can rewrite the above 
equation as:
$$V(t)=\sum_{i=2}^N \sum_{j=1}^{i-1} \frac{1}{r_{i j}}$$
Since in our scenario energy is not conserved (kinetic energy is dissipated via the velocity-dependent
retarding force), we expect the potential to trend towards a minimum as $t \rightarrow \infty$.

\subsubsection*{Equivalence Classes}

The concept of \textit{equivalence classes} is introduced in order to characterize the different
equilibrium configurations of the point charges on the unit sphere. As $t \rightarrow \infty$ and 
$V$ is minimized, the equilibrium configuration for $N$ charges becomes independent of the initial
conditions. In words, the number of equivalence classes is the number of groups of charges that are 
indistinguishable in the equilibrium configuration. 

In order to calculate the number of equivalence classes, the magnitude of the displacement vector 
from charge $i$ to charge $j$ is defined as 
$$d_{i j}=\left|\mathbf{r}_j-\mathbf{r}_i\right| \quad i, j=1,2, \ldots, N$$
For charges $i$ and $i'$ in the same equivalence class, the lists of magnitudes $d_{i j}$ and 
$d_{i' j}$ to every other charge $j$ are the same.

%%%%%%%%%%%%%%%%%%%%%%%%%%%%%%%%%%%%%%%%%%%%%%%%%%%%%%%%%%%%%%%%%%%%%%%%%%%%%%%%%%%%%%%%%%%%%%%%%%%%%%%
% Numerical approach
%%%%%%%%%%%%%%%%%%%%%%%%%%%%%%%%%%%%%%%%%%%%%%%%%%%%%%%%%%%%%%%%%%%%%%%%%%%%%%%%%%%%%%%%%%%%%%%%%%%%%%%
\subsection*{Numerical Approach}

\subsubsection*{Finite Difference Equations}

% Include normalization of positions 

\subsubsection*{Convergence Testing}


%%%%%%%%%%%%%%%%%%%%%%%%%%%%%%%%%%%%%%%%%%%%%%%%%%%%%%%%%%%%%%%%%%%%%%%%%%%%%%%%%%%%%%%%%%%%%%%%%%%%%%%
% Implementation
%%%%%%%%%%%%%%%%%%%%%%%%%%%%%%%%%%%%%%%%%%%%%%%%%%%%%%%%%%%%%%%%%%%%%%%%%%%%%%%%%%%%%%%%%%%%%%%%%%%%%%%
\subsection*{Implementation}

% Refer to appendix for finite difference and potential part of charges.m code 

% Equivalence classes code description

% initial random locations of charges

% Describe each of convtest.m, plotv.m, survey.m


%%%%%%%%%%%%%%%%%%%%%%%%%%%%%%%%%%%%%%%%%%%%%%%%%%%%%%%%%%%%%%%%%%%%%%%%%%%%%%%%%%%%%%%%%%%%%%%%%%%%%%%
% Results
%%%%%%%%%%%%%%%%%%%%%%%%%%%%%%%%%%%%%%%%%%%%%%%%%%%%%%%%%%%%%%%%%%%%%%%%%%%%%%%%%%%%%%%%%%%%%%%%%%%%%%%
\subsection*{Results}

\subsubsection*{convtest.m output}

\subsubsection*{plotv.m output}

\subsubsection*{survey.m output}

\subsubsection*{Video of sample evolution}

\pagebreak

%%%%%%%%%%%%%%%%%%%%%%%%%%%%%%%%%%%%%%%%%%%%%%%%%%%%%%%%%%%%%%%%%%%%%%%%%%%%%%%%%%%%%%%%%%%%%%%%%%%%%%%
% Conclusions
%%%%%%%%%%%%%%%%%%%%%%%%%%%%%%%%%%%%%%%%%%%%%%%%%%%%%%%%%%%%%%%%%%%%%%%%%%%%%%%%%%%%%%%%%%%%%%%%%%%%%%%
\subsection*{Conclusions}


\pagebreak

%%%%%%%%%%%%%%%%%%%%%%%%%%%%%%%%%%%%%%%%%%%%%%%%%%%%%%%%%%%%%%%%%%%%%%%%%%%%%%%%%%%%%%%%%%%%%%%%%%%%%%%
% Appendix A - charges.m Code
%%%%%%%%%%%%%%%%%%%%%%%%%%%%%%%%%%%%%%%%%%%%%%%%%%%%%%%%%%%%%%%%%%%%%%%%%%%%%%%%%%%%%%%%%%%%%%%%%%%%%%%
\subsection*{Appendix A - charges.m Code}
\lstinputlisting{../src/charges.m}

\pagebreak

%%%%%%%%%%%%%%%%%%%%%%%%%%%%%%%%%%%%%%%%%%%%%%%%%%%%%%%%%%%%%%%%%%%%%%%%%%%%%%%%%%%%%%%%%%%%%%%%%%%%%%%
% Appendix B - convtest.m Code
%%%%%%%%%%%%%%%%%%%%%%%%%%%%%%%%%%%%%%%%%%%%%%%%%%%%%%%%%%%%%%%%%%%%%%%%%%%%%%%%%%%%%%%%%%%%%%%%%%%%%%%
\subsection*{Appendix B - convtest.m Code}
\lstinputlisting{../src/convtest.m}

\pagebreak

%%%%%%%%%%%%%%%%%%%%%%%%%%%%%%%%%%%%%%%%%%%%%%%%%%%%%%%%%%%%%%%%%%%%%%%%%%%%%%%%%%%%%%%%%%%%%%%%%%%%%%%
% Appendix C - plotv.m Code
%%%%%%%%%%%%%%%%%%%%%%%%%%%%%%%%%%%%%%%%%%%%%%%%%%%%%%%%%%%%%%%%%%%%%%%%%%%%%%%%%%%%%%%%%%%%%%%%%%%%%%%
\subsection*{Appendix C - plotv.m Code}
\lstinputlisting{../src/plotv.m}

\pagebreak

%%%%%%%%%%%%%%%%%%%%%%%%%%%%%%%%%%%%%%%%%%%%%%%%%%%%%%%%%%%%%%%%%%%%%%%%%%%%%%%%%%%%%%%%%%%%%%%%%%%%%%%
% Appendix D - survey.m Code
%%%%%%%%%%%%%%%%%%%%%%%%%%%%%%%%%%%%%%%%%%%%%%%%%%%%%%%%%%%%%%%%%%%%%%%%%%%%%%%%%%%%%%%%%%%%%%%%%%%%%%%
\subsection*{Appendix D - survey.m Code}
\lstinputlisting{../src/survey.m}

\pagebreak

\end{document}